\documentclass[11pt]{article}

\usepackage{graphicx,amsmath,amssymb,url,xspace}
\usepackage[utf8]{inputenc}
\usepackage[colorlinks=true]{hyperref}
\newcommand{\eg}{e.g.,\xspace}
\newcommand{\bigeg}{E.g.,\xspace}
\newcommand{\etal}{\textit{et~al.\xspace}}
\newcommand{\etc}{etc.\@\xspace}
\newcommand{\ie}{i.e.,\xspace}
\newcommand{\bigie}{I.e.,\xspace}

\title{Technical Note}
\author{Stian Hvatum and Benjamin Bjørnseth}

\begin{document}
\maketitle

\section{Task 1}
\label{sec:task1}
We need 12 states to implement the error tagging, since any
combination of the microcontrollers can fail. Since three or more
failures indicate a broken system, we can merge the states
corresponding to these scenarios into one ``System broken'' state,
which means that we do not need the full 16 states. However, since 3
bits are not enough to represent 12 states, we need 4 bits anyways.

From the initial state, you can move to the states a, b, c, d and
System broken, if either microcontroller a, b, c, d fail or two
microcontrollers fail simultaneously (in which case you do not know
which result to trust, and the system has to be considered
broken). From the states a, b, c and d, you can move to states
representing two microcontrollers having failed. For instance, from
state a you can move to state ab, ac or ad if b, c or d fails,
respectively. From any state you can move back to the Initial state if
the reset signal state is set.

\section{Task 2}
\label{sec:task2}

In the following evaluation of our solutions, we will consider the
following critera:
\begin{itemize}

%Evaluation criteria
\item Area
\item Speed
\item Genericity
\item Readability
\item Transferrability to ASIC?

\end{itemize}

In our group, we considered four different implementations. 

The first implementation represented the states discussed in
\autoref{sec:task1} explicitly with a custom VHDL state type, and had
a case statement over a state variable which implemented the logic of
each case. This resulted in approximately 300 lines of VHDL code, and
synthesized into 98 LUTs with a maximum cloock frequency of 291.3
MHz. It was, as such, hard to read, gigantic, and overall a horrendous
design. Nevertheless, we were taught a bit about how NOT to write VHDL
programs.

The second implementation tried to alleviate this by writing the logic
outside of the sequential process with select statements for
determining the voted output, the status and the next internal
state. In this implementation (as well as the next two), the state was
encoded with a four-bit vector instead of using a state type. This
implementation was significantly shorter, and quite simple to read. It
yielded 19 LUTs, and a clock speed of 280 MHz. It was, however, not
generic in the sense that adding a fifth microcontroller would require
doubling the cases of the different select statements. 

The third implementation used a clever algorithm for performing the
voting, based on summing the number of valid signals with each value
and checking which value was highest, (\ie sum(validOnes) >
sum(validZeroes)). It also used summations over the number of working
signals (\ie our state) to calculate the status output signal. This
design gave us 19 LUTs and a clock frequency of 300 MHz, and was as
such similar to the previous solution. The previous solution computed
the status output signals by using three LUTs, converting the four-bit
state signal into each bit of the status. This solution used more due
to the addition circuitry being more complex as well as also requiring
three LUTs for converting the sum into each of the status bits. When
we combined these two designs (\ie using the sum algorithm for voting
and the plain LUT approach for calculating the status and new internal
state), we ended up with 14 LUTs and a clock frequency of 320 MHz. 

The final implementation took the approach of coding the desired LUTs
of the voter explicitly, taking advantage of structures intrinsic to
the problem. By hand-coding this, we were able to get the voting logic
down to using only 3 LUTs, which resulted in a total circuit size of
10 LUTs. The speed of the circuit was 360 MHz, so this design is
superior in these aspects. It is, however, somewhat harder to read, as
the correct functioning of the expressions in the voting logic is not
necessarily obvious. In the 

Tradeoffs:

\begin{itemize}

%Evaluation criteria
\item Area
\item Speed
\item Genericity
\item Readability
\item Transferrability to ASIC?

%Solutions:
\item Calculating the error tags concurrently with the voted output or
  based on the voted output.
\item Algorithm for voting:
  - Sum the numbers
  - Using LUTs explicitly, taking care of implicit ``this just works''-cases 
  - Simple case-statement
  - Explicit state machine
\item Output -  count versus mux
  - One bit for each, work/fail
  - 
  

\end{itemize}

\section{Task 3}
\label{sec:task3}



\end{document}


\section{Modules overview}
More information about the modules can be found in \autoref{apx:DS} and \ref{apx:HW_DN}. 

\subsection{Front-end interface}
The front-end interface\index{Front-end interface}(FEI) is our way of implementing the input amplifier\index{Input amplifier} and analog-to-digital converter\index{ADC}. The FEI \nomenclature{FEI}{Front-end interface} also implements a phantom power\index{phantom power} supply for the external microphone, following the 48V standard for phantom power. \\
When designing the FEI a few specifications are important to consider: 

The dynamic range\index{dynamic range} (the difference between the strongest and the weakest signal detectable) of human hearing\index{human hearing} is very high. This allows us to hear very faint sounds. Therefore, the FEI\index{Front-end interface} must have a very low noise\index{noise} output, in order for the user experience to be pleasant. The solution is to make a low noise amplifier and use a high resolution ADC\index{ADC}. 

The FEI also has variable gain and clipping detection. The variable gain is implemented in the input amplifier and the clipping detection is a feature of the ADC\index{ADC}, which has digital outputs to indicate clipping\index{clipping}. 

To reduce noise picked up by the cable\index{cable} (microphone lead), the analog\index{analog} signal input is of the differential\index{differential} type. This cancels out noise\index{noise} picked up by both wires (common-mode noise). Using differential\index{differential} signals is also the standard in professional audio equipment. 

In order to comply with other audio equipment, the voltage levels must be the same for a full-scale signal as the standard dictates. This (along with the voltage levels of the ADC\index{ADC}) in turn dictates the gain of the input amplifier. 

The output impedance\index{impedance} of a modern microphone is typically very low (less than 200 ohms). Therefore the input impedance of the FEI\index{Front-end interface} does not have to by very high. A value in the order of 10 kohms will suffice. Keeping the impedance low is desirable, since a higher impedance will produce more noise. 

\subsection{Output interface}
The output interface (OI) is our way of implementing the digital-to-analog converter\index{DAC} and output amplifier\index{Output amplifier}. For the same reasons of human hearing\index{human hearing} characteristics as for the front-end interface\index{Front-end interface}, the OI\index{Output interface} must have low noise amplifiers and a high resolution DAC\index{DAC}.\nomenclature{DAC}{Digital to analog converter}  

The bandwidth of the OI\nomenclature{OI}{Output interface}  must be the same as the whole system. The OI does not need a precise high pass filter, only AC\nomenclature{AC}{Alternating current}  coupling capacitors able to pass a signal of low enough frequency.  

\subsection{DSP module} \index{DSP}
Between the front-end interface\index{Front-end interface} and the output interface, there is a digital signal processing module that handles the feedback detection and the sound filtering.
In order to avoid to much delay in this unit, the feedback detection module must copy a chunk of the sound stream and process it without blocking the stream. This makes the
system asynchronous, and there will be a short delay from the howling is present until it has been detected.

The original sound stream\index{sound stream} is sent through the filtering unit which removes those frequencies signaled by the feedback detection module. This way, the feedback detection
module can use some time analyzing the sound stream without delaying the sound. The sound is then sent on to the output interface\index{Output interface}, where the digital sound stream is
converted to real sound without the howling\index{howling} frequencies.

Our idea was to use an FPGA\index{FPGA}\nomenclature{FPGA}{Field-programmable gate array}  and its DSP-slices\index{DSP-slices} to implement this functionality. There are plenty of example code the Internet on how to implement both Notch filtering\index{Notch-filter} and 
Fast Fourier Transform \index{FFT}\nomenclature{FFT}{Fast Fourier Transform} on an FPGA, which would make this approach a lot easier than if we had to do everything from scratch.

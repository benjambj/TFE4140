In this project, we have implemented the Liaison System, a system that assures a receiver that the results are as correct as possible.
The system takes input from four equal but distinct microcontrollers and provides a layer of data consistency. The Liaison keeps track
of faulty microcontrollers, and makes sure that no faulty microcontroller are allowed to send its result. The Liaison also adds an error
correcting code using (15,11)-Hamming code with SECDED, which allows the receiver to correct a single error and detect up to two errors.

The project has given us a wider insight in the world of VHDL and hardware design. We have used a significant amount on both testing and
on area optimization, which both counts as important experience. The end result was a circuit with a decent area usage compared to the
expected results given by the cource staff, and we are quite happy with the outcome. We do believe, however, that there are still room for
improvements, as there are parts of the VHDL code that was not tweaked extensively. Possible simplifications of the current system includes
parallel calculation of the ECC bits as described in \autoref{sec:alternativeoveralldesign} might have saved us a few LUTs, but added a few
registers. In addition, we had a selection algorithm that used only 3 LUTs, but there was a bug we were unable to fix. However, we believe
that with some more tinkering, a sub 30 LUT solution exists.

Our project was tested using over 260 000 test vectors, which should cover most of the interesting cases, but is still not system wide
exhaustive. As the course also introduced formal verification, we think that a more advanced approach would utilized the formal methods.

All in all, this project has been informative, challenging and fun.

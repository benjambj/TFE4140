\subsection{Lifetime Expectancy Calculations}
To perform the required lifetime expectancy calculations, some
assumptions about the stochastic behaviour of the microcontrollers are
made. An assumption that was provided by the problem text, was that
the expected lifetime of a single microcontroller was six years. In
addition to this, we assumed that the lifetime of each microcontroller
followed an exponential probability distribution, and that their
individual lifetimes are independent. 

The first assumption is made because the exponential distribution is
simple to use for calculations. Even though it is not suitable for
describing the lifetime of an entire technical device, it is used
extensively in reliability theory\cite{wikipedia}. Thus, it should
providing an estimate of how the system behaves. It was also suggested
by the course staff.

The second assumption is also somewhat optimistic, since the
microcontrollers are exposed to the same environmental hazards at the
same time. For instance, a solar flare will most likely strike all the
microcontrollers simultaneously, and thus microcontroller lifetimes
are likely to have a non-zero covariance. The assumption is still made
to make the calculations easier. If a more accurate model is
desirable, dependence should be taken into account.

\begin{align*}
Q(4) + Q(3) &= \binom{4}{4}P(t)^4 + \binom{4}{3}P(t)^3(1 - P(t)) \\
            &= (e^{\frac{t}{6}})^4  + 4 (e^{\frac

}
\end{align*}

\subsection{Possible Simplifications}
Overall parity bit simplification?

Assumption that microcontroller data can be used for error tagging
even outside of the di\_ready+8cycles intervals?

\subsection{Alternative Solutions}
- Different voting algorithms
- Simultaneous calculation of error tags and voted data (is it on critical path?)
- Store data in register and calculate ECC from these instead of on the fly?
- More as described in technote?

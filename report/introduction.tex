\subsection{Background}
Digital machinery launched into space is exposed to a number of
environmental hazards not normally encountered in operation on
earth. This includes dangers such as solar flares, cosmic background
radiation etc.\cite{nasa}

One way of coping with these problems is to introduce redundancy in
the system design by duplicating elements. For instance, instead of
using a single microprocessor to perform relevant calculations, one
might use four of them. This way the system can continue its operation
even in the face of failure in one or possibly even several of the
redundant components.

For a component to be of use, it will naturally have to either produce
some result to be consumed by another part of the system or change the
state of the system. Duplicating interal components of the system will
therefore have an impact on their surroundings, since communicating
entities will only accept data from a single source. It is also
undesirable to allow components which have failed to change the system
state.

There are several ways to tackle this design issue. One would be to
introduce redundancy in the components which communicates with the
duplicated component---if you duplicated a sensor, for instance, you
could also duplicate the processors it communicates with. Another
strategy would be to add another components which combines the output
from the duplicated components and sends this as input to the
peripheral unit.

\subsection{Description of problem} \todo{Or Problem Statement?}  

Our task was to design a system component, named the Liaison System,
which handled this problem by taking the latter
approach. Specifically, the Liaison System was to work in an
environment containing four microprocessor performing the same
computation. This environment is depicted in
\autoref{fig:systemenvironment}. 

\missingfigure{Figure of the Liaison system environment}
\label{fig:systemenvironment}

These microprocessors emit eight bit data words serially. The task of
the Liaison System was to perform a majority vote amongst the
microcontrollers assumed to work, and tag microcontrollers as broken
if their data differed from the majority.

The Liaison should also emit a status code after each majority eight
bit word, describing what failures it had observed in the system up to
this point. The status codes are listed in \autoref{tab:systemstatus}.

\begin{table}[htbp]
  \centering
  \begin{tabular}{|c|c|}
    \hline \\
    \textbf{Failed Microcontrollers} & \textbf{Binary Status Code} \\ \hline
    0 & 000 \\ \hline
    1 & 001 \\ \hline
    2 & 010 \\ \hline
    3 or 4 & 111 \\ \hline
  \end{tabular}
  \caption{System status code given microcontroller failures.}
  \label{tab:systemstatus}
\end{table}

In addition to this, the Liaison should also be able to generate an
error correcting code which could be used to detect and repair a one
bit error and detect a two bit error in the data word and status. We
were required to use a Hamming code.\todo{Reference?}

It was also stated that the design should focus on getting a small an
implementation of this as possible.

Finally, in addition to designing the Liaison system, we should also
perform an evaluation of its expected lifetime.

\subsection{Description of solution}\todo{Skal dette være med?}

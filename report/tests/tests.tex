The liaison system was verified utilizing automated testbenches. The testbenches uses VHDL assertions to notify the author if any
of the tests failed. In order to create a good and short but extensive testbench, we have built a collection of procedures that takes input, expected output
and expected system status after all input has been sent.

The test cases was divided into three blocks.
\begin{itemize}
\item The first test block consists of 8 tests that asserts the normal behaviour. It is done with some easy cases and a 2 cycle delay between inputs.

\item The second test block consists of 24 cases that at some point broke our logic.  tests where our design was likly to have flaws. During development, these tests were
triggered often, proving their value.

\item The third and last test block is a permutation of working state and all 4! fail states, each with all MCUs sending
all possible inputs (all MCUs sends the same value to keep the number of tests down to an affordable number). This block consists of 16 836 different test cases,
and cover almost any possible state. The states that are not covered are those where one of the MCUs fails in the middle of the transaction, which are covered in
the second block.
\end{itemize}

Since all permutations of fail states and all permutations of valid input within these states
were applied to the circuit during simulation, we are confident that the Liaison System is working correctly. We have also tested the circuit with a post-place and route
simulation verifying that the output is also correct with timing taken into account\footnote{The timing simulation was done with iSim from Xilinx, synthesised with XST. Both tools are part of the ISE Design Suite}.

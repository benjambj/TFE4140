\documentclass[titlepage,12pt,a4paper]{article}

\usepackage[pdftex]{graphicx}
%\usepackage[symbol]{footmisc}
\usepackage[utf8]{inputenc}

\usepackage{tikz}
\usetikzlibrary{shapes,decorations}
\usetikzlibrary{dsp,chains}

\usepackage{makeidx}
%\usepackage{showidx}
\usepackage{parskip}
\usepackage{float}
\usepackage{calc}
\usepackage{pdfpages}
%\usepackage[textwidth=2.3213cm]{todonotes}
\usepackage{amsfonts}
\usepackage{amsmath}
\usepackage{pifont}
\usepackage{ulem}
\usepackage{pdflscape}
\usepackage{geometry}
\usepackage{listings}
\usepackage{color}
\usepackage{textcomp}

\definecolor{listinggray}{gray}{0.9}
\definecolor{lbcolor}{rgb}{0.9,0.9,0.9}
\lstset{
keywordstyle=\bfseries\ttfamily\color[rgb]{0,0,1},
identifierstyle=\ttfamily,
commentstyle=\color[rgb]{0.133,0.545,0.133},
stringstyle=\ttfamily\color[rgb]{0.627,0.126,0.941},
showstringspaces=false,
    basicstyle=\scriptsize,
    numberstyle=\footnotesize,
    framexleftmargin=3pt,
    numbers=left,
    stepnumber=1,
    numbersep=15pt,
    tabsize=2,
    breaklines=true,
    prebreak = \raisebox{0ex}[0ex][0ex]{\ensuremath{\hookleftarrow}},
    breakatwhitespace=false,
    aboveskip={1.5\baselineskip},
    columns=fixed,
    upquote=true,
    extendedchars=true,
    frame=TLBR,
    backgroundcolor=\color{lbcolor},
}

\usepackage[backref=page]{hyperref}
\def\backref#1{{\scriptsize [#1]}}

\newcommand{\HRule}{\rule{\linewidth}{0.35mm}}

\def\theaipage{\string\hyperpage{\thepage}}

\hypersetup{
    bookmarks=true,         % show bookmarks bar?
    unicode=false,          % non-Latin characters in Acrobats bookmarks
    pdftoolbar=true,        % show Acrobats toolbar?
    pdfmenubar=true,        % show Acrobats menu?T
    pdffitwindow=false,     % window fit to page when opened
    pdfstartview={FitH},    % fits the width of the page to the window
    pdftitle={Design of Liaison System, TFE4140 Term Project},    % title
    pdfauthor={Stian Hvatum, Benjamin Bjørnseth},     % author
    pdfsubject={TFE4140 Term Project},   % subject of the document
    pdfcreator={Stian Hvatum, Benjamin Bjørnseth},   % creator of the document
    pdfproducer={Stian Hvatum, Benjamin Bjørnseth}, % producer of the document
    pdfnewwindow=true,      % links in new window
    colorlinks,       % false: boxed links; true: colored links
    linkcolor=black,          % color of internal links
    citecolor=blue,        % color of links to bibliography
    filecolor=magenta,      % color of file links
    urlcolor=cyan           % color of external links
}




% to change colors
\newcommand{\bordercol}{red}

%% code by Andrew Stacey 
% http://tex.stackexchange.com/questions/51582/background-coloring-with-overlay-specification-in-algorithm2e-beamer-package#51582

\makeatletter
\tikzset{%
     remember picture with id/.style={%
       remember picture,
       overlay,
       save picture id=#1,
     },
     save picture id/.code={%
       \edef\pgf@temp{#1}%
       \immediate\write\pgfutil@auxout{%
         \noexpand\savepointas{\pgf@temp}{\pgfpictureid}}%
     },
     if picture id/.code args={#1#2#3}{%
       \@ifundefined{save@pt@#1}{%
         \pgfkeysalso{#3}%
       }{
         \pgfkeysalso{#2}%
       }
     }
   }

   \def\savepointas#1#2{%
  \expandafter\gdef\csname save@pt@#1\endcsname{#2}%
}

\def\tmk@labeldef#1,#2\@nil{%
  \def\tmk@label{#1}%
  \def\tmk@def{#2}%
}

\tikzdeclarecoordinatesystem{pic}{%
  \pgfutil@in@,{#1}%
  \ifpgfutil@in@%
    \tmk@labeldef#1\@nil
  \else
    \tmk@labeldef#1,(0pt,0pt)\@nil
  \fi
  \@ifundefined{save@pt@\tmk@label}{%
    \tikz@scan@one@point\pgfutil@firstofone\tmk@def
  }{%
  \pgfsys@getposition{\csname save@pt@\tmk@label\endcsname}\save@orig@pic%
  \pgfsys@getposition{\pgfpictureid}\save@this@pic%
  \pgf@process{\pgfpointorigin\save@this@pic}%
  \pgf@xa=\pgf@x
  \pgf@ya=\pgf@y
  \pgf@process{\pgfpointorigin\save@orig@pic}%
  \advance\pgf@x by -\pgf@xa
  \advance\pgf@y by -\pgf@ya
  }%
}
\makeatother

\newcommand{\tikzmarkin}[2][]{%
      \tikz[remember picture,overlay,baseline=1ex]
      \draw[double,line width=1pt,#1]
      (pic cs:#2) -- (0,0);}


\newcommand\tikzmarkend[2][]{%
\tikz[remember picture with id=#2,baseline=1ex] #1;}






\begin{document}

\begin{titlepage}

\begin{center}

    %    \includegraphics[scale=0.7]{./figures/Fortitudo_floris}\\[1cm]

\textsc{\LARGE TTK4850 - EiT Byggelandsbyen}\\[0.2cm]

\HRule \\[0.5cm]

{\huge Miksebord}\\[0.2cm]
\small{Project report}\\[1cm]

\begin{table}[h]
\centering
\begin{tabular}{rl}
Thomas Hasfjord   &  Bård Bakken Stovner\\
Andreas Eilertsen & Håkon Furre Amundsen\\
\multicolumn{2}{c}{Stian Hvatum}
\end{tabular}
\end{table}

\vfill
\large{\today}

\end{center}

\end{titlepage}

\pagenumbering{Roman}
This report contains our solution to the term project in the course
TFE4140. We describe our design of a liaison system implemented on an
FPGA, aimed at performing a majority vote amongst microcontrollers
transmitting data words serially. The main focus of our implementation
was to keep the area usage as low as possible, measured in the number
of LUTs used on the FPGA. We discuss alternative solutions, and
examine their relative strenghts and weaknesses.  Our proposed
solution uses 31 LUTs, and runs with a clock frequency of 298.7 MHz.


\newpage
\tableofcontents
\listoffigures
\listoftables

\newpage
\pagenumbering{arabic}
%sections goes here, use \input{file[.tex]}

\section{Introduction}
\subsection{Background}
Digital machinery launched into space is exposed to a number of
environmental hazards not normally encountered in operation on
earth. This includes dangers such as solar flares, cosmic background
radiation etc.\cite{nasa}

One way of coping with these problems is to introduce redundancy in
the system design by duplicating elements. For instance, instead of
using a single microprocessor to perform relevant calculations, one
might use four of them. This way the system can continue its operation
even in the face of failure in one or possibly even several of the
redundant components.

For a component to be of use, it will naturally have to either produce
some result to be consumed by another part of the system or change the
state of the system. Duplicating interal components of the system will
therefore have an impact on their surroundings, since communicating
entities will only accept data from a single source. It is also
undesirable to allow components which have failed to change the system
state.

There are several ways to tackle this design issue. One would be to
introduce redundancy in the components which communicates with the
duplicated component---if you duplicated a sensor, for instance, you
could also duplicate the processors it communicates with. Another
strategy would be to add another components which combines the output
from the duplicated components and sends this as input to the
peripheral unit.

\subsection{Description of problem} \todo{Or Problem Statement?}  

Our task was to design a system component, named the Liaison System,
which handled this problem by taking the latter
approach. Specifically, the Liaison System was to work in an
environment containing four microprocessor performing the same
computation. This environment is depicted in
\autoref{fig:systemenvironment}. 

\missingfigure{Figure of the Liaison system environment}
\label{fig:systemenvironment}

These microprocessors emit eight bit data words serially. The task of
the Liaison System was to perform a majority vote amongst the
microcontrollers assumed to work, and tag microcontrollers as broken
if their data differed from the majority.

The Liaison should also emit a status code after each majority eight
bit word, describing what failures it had observed in the system up to
this point. The status codes are listed in \autoref{tab:systemstatus}.

\begin{table}[htbp]
  \centering
  \begin{tabular}{|c|c|}
    \hline \\
    \textbf{Failed Microcontrollers} & \textbf{Binary Status Code} \\ \hline
    0 & 000 \\ \hline
    1 & 001 \\ \hline
    2 & 010 \\ \hline
    3 or 4 & 111 \\ \hline
  \end{tabular}
  \caption{System status code given microcontroller failures.}
  \label{tab:systemstatus}
\end{table}

In addition to this, the Liaison should also be able to generate an
error correcting code which could be used to detect and repair a one
bit error and detect a two bit error in the data word and status. We
were required to use a Hamming code.\todo{Reference?}

It was also stated that the design should focus on getting a small an
implementation of this as possible.

Finally, in addition to designing the Liaison system, we should also
perform an evaluation of its expected lifetime.

\subsection{Description of solution}\todo{Skal dette være med?}

\section{Design}
\label{sec:design}
The Liaison System is created to assure that the correct data collected from a microcontrollers is received correctly by the end user.
To achieve this, the system uses four equal but distinct microcontrollers, an internal voting system to assure that if any of the microcontrollers
disagree, it is marked as malfunctioning and is no longer allowed to send output. The output sent from the Liaison is extended with a system status
code that tells the end user if any of the microcontrollers are damaged, and the signal is also enhanced with an error correcting code such that
we can be more certain that the signal is not distorted on its way to the reciever.

\begin{figure}
\includegraphics[width=15cm]{design/fig_overview}
\caption{Modules overview}
\label{fig:overview}
\end{figure}

As we can see from \autoref{fig:overview}, the internals of the Liaison can be modeled as four different pieces of hardware, each providing a
nessesary service to the system.

\subsection{Voting algorithm}
\subsection{State maintainance}
\subsection{Error Correcting Code Generator}
\subsection{Output mulitplexer}

\section{Implementation}
\label{sec:implementation}

\begin{figure}
\includegraphics[width=15cm]{implementation/fig_ecc}
\caption{Implementation of Hamming(16,11)}
\label{fig:ecc}
\end{figure}

In this section, we describe how the module designs detailed in
\autoref{sec:design} are implemented on the FPGA. 

\subsection{Technology Schematic}
\label{sec:technologyschematic}

Figure \autoref{fig:technologyschematic} shows how each functionality of
the Liaison from \autoref{fig:overview} is mapped to the LUTs in the syntesised design.
Each functionality are marked with a distinct colour. Note that LUTs marked as either
``Status Calculation'' or ``State Maintenance'' functionality are both a part of the
State Maintenance module.


\begin{figure}[p]
  \vspace*{-1.2in}
  \centerline{ \includegraphics[width=1.2\textwidth]{LUT-count} }
  \caption{The technology schematic of our Liaison}
  \label{fig:technologyschematic}
\end{figure}

\subsection{Voting Algorithm Implementation}
Our design 

\subsection{State Maintenance Implementation}

\subsection{Error Correction Code Implementation}

\subsection{Output Selection Implementation}

\subsection{Results}

\todo[inline]{Describe LUT usage, register usage and clock frequency}

\section{Verification}
\subsection{Overview}
The liaison system was verified utilizing automated testbenches. The testbenches uses VHDL assertions to notify the author if any
of the tests failed. In order to create a good and short but extensive testbench, we have built a collection of procedures that takes input, expected output
and expected system status after all input has been sent.

The test cases was divided into three blocks.
\begin{itemize}
\item The first test block consists of 8 tests that asserts the normal behaviour. It is done with some easy cases and a 2 cycle delay between inputs.

\item The second test block consists of 24 cases that at some point broke our logic.  tests where our design was likly to have flaws. During development, these tests were
triggered often, proving their value.

\item The third and last test block is a permutation of working state and all 4! fail states, each with all MCUs sending
all possible inputs (all MCUs sends the same value to keep the number of tests down to an affordable number). This block consists of 16 836 different test cases,
and cover almost any possible state. The states that are not covered are those where one of the MCUs fails in the middle of the transaction, which are covered in
the second block.
\end{itemize}

Since all permutations of fail states and all permutations of valid input within these states
were applied to the circuit during simulation, we are confident that the Liaison System is working correctly. We have also tested the circuit with a post-place and route
simulation verifying that the output is also correct with timing taken into account\footnote{The timing simulation was done with iSim from Xilinx, synthesised with XST. Both tools are part of the ISE Design Suite}.

\subsection{Conformance with requirements specification}
This section explain the conformance with each point in the system requirements specification.
\begin{enumerate}
\item{\textbf{The Liaison Interface}} \hfill\\
    Our implementation of the Liaison System follows the signaling interface given by the requirements\cite{task}.

\item{\textbf{Voting}} \hfill\\
    The Voting algorithm has been tested and confirmed to be working. This was done by enumerating all
    permutations of input and state, and then check the Voters result against expected output using
    VHDL assertions.

\item{\textbf{Error tagging}} \hfill\\
    The error tagging is not specified in the requirements, but is nessesary in our implementation to
    achieve correct System Status bits and correct Voting. The error tagging was tested with the same
    tests as the Voting algorithm and does conform with the expected behaviour.

\item{\textbf{System status}} \hfill\\
    The System Status is needed to tell the receiver about the wellness of the microcontrollers. Since
    there exists a direct mapping between the Error tagging and the System Status, this is also
    included in the same tests as Error tagging and Voting.

\item{\textbf{Error correcting code}} \hfill\\
    The Error correcting code is a specific part of the requirements\cite{task}. The module was tested by writing
    a testbench function that generates (15,11)-Hamming Code with SECDED from an 11-bit word. This
    procedure was added to the enumerated tests from the Voter-test such that for each input, the testbench
    would automatically expect a specific ECC. If the ECC from the function was not equal to the output
    of the Liaison, assertions were raised to the tester. Since the soft function agrees with the simulated
    hardware, we are certain that the ECC-module works as specified in the requirements. The ECC-generating
    software function can be found in \autoref{apx:ecc}.

\item{\textbf{Reset behaviour}} \hfill\\
    On reset, The Liaison is set back to inital state on next clock cycle. The reset is stricly synchronous,
    as stated in the requirements\cite{task}, and is thus only read on rising edge of the clock. We have
    tested that reset works for any final state when a data sequence finished, but we have not tested the
    behavoiur of a reset in the middle of a transmission. By simple code inspection, we see that the
    reset is not aware of current system status, but simply sets all the state variables back to initial
    state. That means that after a reset signal is asserted along with rising edge of the clock, the system
    will not process data until a new {\ttfamily di\_ready}.

\item{\textbf{System consitency}} \hfill\\
    Taken into account all the tests ran on the Voter, ECC and Status modules, we are sure that the system works consistent
    with the requirements. Nearly all enumerations of internal state has been verified. We assume that those states that has
    not been explicitly tested work correctly, as their internals don't differ very much. Examples of this are wherever the MCUs
    fail at first bit or another bit within the word. We test only for first-bit failures in the exhaustive test, but both the
    voter and the \todo{write ecc status stuff}

\item{\textbf{New data after $11+m$ cycles}} \hfill\\
    Most of the test cases send data directly after previous data word was correctly received, so 
    Our implementation of the Liaison System follows the signaling interface given by the requirements.

\end{enumerate}

\section{Discussion}
\label{sec:discussion}
\subsection{Lifetime Expectancy Calculations}
To perform the required lifetime expectancy calculations, some
assumptions about the stochastic behaviour of the microcontrollers are
made. An assumption that was provided by the problem text, was that
the expected lifetime of a single microcontroller was six years. In
addition to this, we assumed that the lifetime of each microcontroller
followed an exponential probability distribution, and that their
individual lifetimes are independent. 

The first assumption is made because the exponential distribution is
simple to use for calculations. Even though it is not suitable for
describing the lifetime of an entire technical device, it is used
extensively in reliability theory\cite{wikipedia}. Thus, it should
providing an estimate of how the system behaves. It was also suggested
by the course staff.

The second assumption is also somewhat optimistic, since the
microcontrollers are exposed to the same environmental hazards at the
same time. For instance, a solar flare will most likely strike all the
microcontrollers simultaneously, and thus microcontroller lifetimes
are likely to have a non-zero covariance. The assumption is still made
to make the calculations easier. If a more accurate model is
desirable, dependence should be taken into account.

With these assumptions, $f(t) = \frac{1}{6}e^{-\frac{1}{6}t}$ is the
probability density function of a failure event for an individual
microcontroller, and $F(t) = 1 - e^{-\frac{1}{6}t}$ is the
corresponding cumulative distribution giving the probability that the
microcontroller has failed within time $t$. Consequently, we can
define $P(t) := 1 - F(t) = e^{-\frac{1}{6}t}$ to be the probability
that the microcontroller has not failed within time $t$. Assuming as
we have that each microcontroller is independent, we can view the
potential of failure within a given time as a Bernoulli trial for each
of the microcontrollers. Consequently, the probability that exactly
$r$ microcontrollers have not failed within time $t$, as a function of
$t$, is given by $Q(r) = \binom{4}{r}P(t)^4(1 - P(t))^{4 - r})$. With
this framework, we can begin answering the questions stated in
\autoref{sec:problem}.

\subsubsection{Calculation of Probability of Error in Maximum One Controller}
\label{sec:errorinmaxone}
The probability that at most one controller has failed is the
probability that either four or three controllers are still
working. Since these two events are independent---there cannot be both
exactly four and exactly three microcontrollers that are still working
at a point in time---the total probability is the sum of the
probability of each event. Hence, the probability of an error in at
most one controller within a given time $t$ is given by

\begin{align*}
    Q(4) + Q(3) &= \binom{4}{4}P(t)^4 + \binom{4}{3}P(t)^3(1 - P(t)) \\
                &= (e^{-\frac{t}{6}})^4  + 4 (e^{-\frac{t}{6}})^3(1 - e^{-\frac{t}{6}}) \\
                &= 
\end{align*}

\subsubsection{Calculation of Probability of Error in Maximum Two Controllers}
\label{sec:errorinmaxtwo}
This probability is the same as the probability that either exactly
four, exactly three or exactly two microcontrollers are still working
by time $t$. By the same logic as given in
\autoref{sec:errorinmaxone}, this is the same as
\begin{align*}
  &Q(4) + Q(3) + Q(2) = \\
  &\binom{4}{4}P(t)^4 + \binom{4}{3}P(t)^3(1 - P(t)) + \binom{4}{2}P(t)^2(1 - P(t))^2 = \\
  &
\end{align*}

This function describes the probability that the system is still alive
within time $t$. For future reference, we shall denote it as $R(t)$.

\subsubsection{Calculation of Probability in At Least Three Controllers}
\label{sec:errorinatleastthree}
At least three controllers being broken within time $t$ is the
complement of the event that at most two controllers have failed
within time $t$. Thus, this probability is simply the negation of the
result derived in \autoref{sec:errorinmaxtwo}, which is
\begin{align*}
  1 - R(t) = 1 - (Q(4) + Q(3) + Q(2)) = 
\end{align*}

\subsubsection{Mean Time to Failure}
The mean time to system failure is the expected value of a stochastic
variable $X$ denoting time until at least three microcontrollers have
failed. Now, $Prob\{X \le t\} = Prob\{\text{System has failed within
  time } t\}$, so the cumulative distribution function of $X$,
$F_X(t)$, is given by the result in
\autoref{sec:errorinatleastthree}. If we denote the probability
density function of $X$ by $f_X(t) = \frac{\partial}{\partial
  t}F_x(t)$, we get that

\begin{align*}
  E[X] &= \int_0^{\infty}tf_X(t)dt = \int_0^{\infty}t\frac{\partial}{\partial t} F_X(t)dt \\
       &= \lim_{b \to \infty}[tF_X(t)]_{0}^b - \int_0^{\infty}F_X(t)dt \\
       &= \lim_{b \to \infty}[tF_X(t)]_0^b - \int_0^{\infty}1 - R(t)dt \\
       &= \lim_{b \to \infty}[tF_X(t)]_0^b - \int_0^\infty1dt + \int_0^{\infty}R(t)dt \\
       &= \lim_{b \to \infty}[tF_X(t)]_0^b - \lim_{b \to \infty}[t]_0^b + \int_0^{\infty}R(t)dt \\
       &= \lim_{b \to \infty}(bF_X(b) - 0F_X(0)) - \lim_{b \to \infty}(b - 0) + \int_0^{\infty}R(t)dt \\
       &= \lim_{b \to \infty}bF_X(b) - \lim_{b \to \infty}b + \int_0^{\infty}R(t)dt \\
       &= \lim_{b \to \infty}b\lim_{b \to \infty}F_X(b) - \lim_{b \to \infty}b + \int_0^{\infty}R(t)dt \\
       &= \lim_{b \to \infty}(b - b) + \int_0^{\infty}R(t)dt \\
       &= \int_0^{\infty}R(t)dt
\end{align*}

\todo[inline]{Discuss the MTTF}

%\begin{align*}
%x = 3
%Q(4) + Q(3) &= P(t)^4 + \binom{4}{3}P(t)^3(1 - P(t)) 
%             &= (e^{\frac{t}{6}})^4  + 4 (e^{\frac{t}{6}})^3(1 - e^{\frac{t}{6}})

%%%%%\end{align*}


\subsection{Alternative Solutions}
One of the major goal on this task was to optimize for area useage. In order to achieve the most
efficient solution regarding LUT useage, we came up with three fundamentally different solutions.
Except for the solution described in \autoref{sec:design}, the two 

- Different voting algorithms. Cleaner/faster? Had a 3-LUT one 50 MHz
faster and 3 LUTs smaller with a bug, but with more tinkering it might
be possible to get this to work. 

- Simultaneous calculation of error tags and voted data (is it on
critical path?) Faster?

- Store data in register and calculate ECC from these instead of on
the fly? Cleaner

- More as described in technote?

\subsection{Possible Simplifications of Current Implementation}

\paragraph{SECDED-bit Simplification} \hfill \\
\todo{Reference previous explanation of SECDED bit}. Instead of
performing this XOR operation with the parity bits themselves, one can
exploit the fact that the parity bits are simply an XOR of data bits
to express the overall parity bit just in terms of data bits. To see
this, let $d(i)$ denote data bit $i$ and $p(j)$ denote parity bit $j$.
Then, by substituting the equation for calculating $p(0)$ into the
equation for $p(4)$, we get
\begin{align*}
  p(4) &= d(0) \otimes d(1) \otimes \ldots \otimes d(10) \otimes p(0) \otimes p(1) \otimes p(2) \otimes p(3) \\
  &= d(0) \otimes d(1) \otimes \ldots \otimes d(10) \otimes \\
  & \qquad (d(0) \otimes d(1) \otimes d(3) \otimes d(4) \otimes d(6)
  \otimes d(8) \otimes d(10)) \otimes p(1) \otimes p(2) \otimes p(3) \\
  &= d(2) \otimes d(5) \otimes d(7) \otimes d(9) \otimes p(1) \otimes p(2) \otimes p(3)
\end{align*}

This substitution and simplification could be done with any desirable
combination of parity bits to shift the work between performing XOR
operations with parity bits during the output stage or calculating
state-enable signals during the data parity count accumulation
stage. By experimenting with different combinations, it might be
possible to find a configuration which yields a better
result than the current solution with no simplification does. 

\paragraph{Weaker Assumptions of Microcontroller Data Lines Validity} \hfill \\
In our design, we made the assumption that we should not use the input
data from the microcontrollers to update the error tags outside of
periods in which they were actually sending a word to the Liaison. If
this restriction was relaxed, we would be able to remove a few
registers. The register containing the enable signal, for instance,
would be of no use. It would also be able to shift the responsibility
of making the circuit sequential from the input stage to the output
stage --- instead of using flip-flops for storing the input data from
the microcontrollers, we could use a flip-flop for maintaining the
value of the output signals. Since there would no longer be a need to
avoid updating error tags during a clock period, the prohibiting issue
of enabling the voter for the first data bit would no longer be
present.

\todo[inline]{Is this correct? Could we not have flip flops on output now as well? We would just have to update the enable register in a different way.. but wait, it would have to be set when di\_ready is high!}


\section{Conclusion}
In this project, we have implemented the Liaison System, a system that assures a receiver that the results are as correct as possible.
The system takes input from four equal but distinct microcontrollers and provides a layer of data consistency. The Liaison keeps track
of faulty microcontrollers, and makes sure that no faulty microcontroller are allowed to send its result. The Liaison also adds an error
correcting code using (15,11)-Hamming code with SECDED, which allows the receiver to correct a single error and detect up to two errors.

The project has given us a wider insight in the world of VHDL and hardware design. We have used a significant amount on both testing and
on area optimization, which both counts as important experience. The end result was a circuit with a decent area usage compared to the
expected results given by the cource staff, and we are quite happy with the outcome. Anyways, we believe that there are still room for
improvements, as there are parts of the VHDL code that was not tweeked extensively.

Our project was tested using over 260 000 test vectors, which should cover most of the interesting cases, but is still not system wide
exhaustive. As the course also introduced formal verification, we think that a more advanced approach would utilized the formal methods.

All in all, this project has been informative, challenging and fun.


\newpage 

\appendix
\section{ECC generation for VHDL Testbench}
\label{apx:ecc}
\lstinputlisting[language=VHDL]{apx/ecc.vhd}

\section{Alternative Voting Algorithm: Bit-counter}
\label{apx:bitcnt}
{\ttfamily mcu} is a vector containig the data from the microcontrollers,
and {\ttfamily active} is the inverted error tags.
\lstinputlisting[language=VHDL]{apx/bitcnt.vhd}

\section{Alternative Voting Algortihm: Lookup table}
\label{apx:table}
{\ttfamily fails} represents the error tags.
\lstinputlisting[language=VHDL]{apx/table.vhd}
\section{Liaison interface}
\label{apx:liaison}
This is the interface required on the Liaison System
\lstinputlisting[language=VHDL]{apx/liaison.vhd}

\bibliographystyle{plain}
\bibliography{bibliography}

\end{document}

%Motivation
As previously stated, acoustic \index{acoustic feedback} feedback is a big problem in \index{sound reinforcement systems}sound reinforcement systems. This is normally solved manually by trained sound technician. Sound technicians are expensive to hire, but for small events or shows they are widely not needed unless there is a problem. If howling, one of the biggest problems, could be prevented by an automatic system, there is no need for a technician. The system could also be implemented in modern live mixing desks to provide a useful tool for technicians, so they can pay attention to mixing instead of worrying about feedback.

There exists commercial systems for automaticly removing feedback, but these are not very popular. This could be because they are mostly expensive and difficult to use. In addition, sound technicians are generally perceived to be a conservative group. As an example: it is only a couple of years ago since mixing desks with digital processing where considered ``good enough'' for live use. Sound technicians normally wants complete control of all aspects of the sound mix, so having a component that modifies the sound on its own is widely avoided. It could also be that the implementation of these type of systems are difficult or that the current algorithms used are poorly designed. Causing the system to fail by either not suppressing howling or modifying wanted parts of the sound in a negative way. This projects main goal was to investigate whether a working howling suppression system could be designed with limited resources
